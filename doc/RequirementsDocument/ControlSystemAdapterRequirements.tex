\documentclass[11pt,a4paper]{scrartcl}

%%%%%%%%%%%%%%%%%%%%%%%%%%%%%%%%%%%%%%%%%
%  Default packages
%%%%%%%%%%%%%%%%%%%%%%%%%%%%%%%%%%%%%%%%%
\usepackage{textcomp}%\textmu 
\usepackage{graphicx}
\usepackage{subfig}%allow subfigures
\usepackage{amsmath}%improve layout of formulas
\usepackage{color}
\usepackage{units}%\unitfrac (nice m/s in text and math mode)
\usepackage{url}
\usepackage{booktabs}%\toprule, \midrule, bottomrule in tables
\usepackage{xspace}%automatically add space at the end of macros
\usepackage{upgreek}%\upright greek letters in math environment
\usepackage{authblk}%several authors with different affiliations
\usepackage{threeparttable}% for footnotes within tables
\usepackage{multicol}%multimple columns

%% clickable links in pdf
\usepackage[pdftex,bookmarks=true,breaklinks=true,bookmarksnumbered=true,
            colorlinks=true,linkcolor=webgreen,citecolor=webred,urlcolor=webblue]{hyperref}
%% link colors
\definecolor{webred}{rgb}{0.5, 0, 0}
\definecolor{webgreen}{rgb}{0, 0.5, 0}
\definecolor{webblue}{rgb}{0, 0, 0.5}

%Command to allow footnotes within floats (tables, figures....)
%Use \floatingfootnotemark inside the float, which only places the mark and
%counts up the footnote counter.
%Immediately after the float you put the footnote text. We will see if it
%shows up on the right page....
\newcommand{\floatingfootnotemark}{\addtocounter{footnote}{1} \footnotemark[\value{footnote}]}
\newcommand{\floatingfootnotetext}[1]{\footnotetext[\value{footnote}]{#1}}

%A \fixme command with counter. At the end of the document you can display the counter to see how many FIXMEs are left
\newcounter{nFixmes}
\setcounter{nFixmes}{0}
\newcommand{\fixme}[1]{\addtocounter{nFixmes}{1}\textcolor{red}{(\includegraphics[height=2ex]{fixme} #1)}\xspace}

%file with the svn revision, generated using svnversion
\input{svnrevision.tex}
\usepackage{draftwatermark}
\SetWatermarkScale{3}
\SetWatermarkText{DRAFT \svnshortrev}

\title{Requirement for the\\ ChimeraTK Control System Adapter}
\date{\svnrevision\\ PDF build \today}
\author[1]{Martin Killenberg} \author[2]{Sebastian Marsching}
\author[3]{Adam Poitrokwsi}
\author[1]{Christian Schmidt}

\affil[1]{Deutsches Elektronen-Synchrotron DESY, Hamburg, Germany}
\affil[2]{aquenos GmbH, Baden-Baden, Germany}
\affil[3]{FastLogic Sp.\ z o.\ o., \L\'od\'z, Poland}

%%%%%%%%%%%%%%%%%%%%%%%%%%%%%%%%%%%%%%%%%
% This is where the document really begins
%%%%%%%%%%%%%%%%%%%%%%%%%%%%%%%%%%%%%%%%%
\begin{document}

\maketitle
\section{Objective and Scope}

ChimeraTK's control system interface adapter is intended to decouple the business
logic of a server from the actual control system in use. For this it provides
two functionalities: 
\begin{enumerate}
  \item Control System Variable adapters
  \item Function callbacks to react on control system interrupts
\end{enumerate}

\subsection{Control System Variable adapters}

The control-system variable adapter is a thin layer which provides a
standardised interface for accessing the variable content, but does not provide
control-system functionality. In other words, a control system variable that
holds a signed 32 bit integer can be accessed as 32 bit integer from the
business logic, but a variable history, which might be provided by the control
system, is not accessible. The history is, however, accessible via the control
system. Like this, no control system functionality has to be abstracted or
duplicated/implemented if not available on certain control systems. 

The only exception are two callback functions which can be registered as
actions to be taken when the variable is changed or read. This is needed to allow the
synchronisation with the control system. The business logic can choose between
a set()-function which only informs the control system, or a set()-function
which also triggers the callbacks. When the variable is changed by the control
system, the callbacks are always triggered. A more detailed explanation of the
variable adapter can be found in
section~\ref{section_process_variable_adapter}. 

\subsection{Control System Callback Function}

Control systems can trigger actions on signals, user interrupts or might
implement events which represent an action which is not coupled to a process
variable. The control-system adapter implements an interface so that the
business logic can define functions to be executed, and the control-system
dependent code implements how the particular function is triggered by the
specific control system. 

\subsection{Python interface}

For easy interactive testing, it would be nice to have a python interface to
the control system stub. Like this one can run tests of the full business
logic without the necessity to install a control system. The interface might
also be used to implement tests. 

\section{Data Types}

\subsection{Simple Data Types}

The control system interface adapter supports ProcessVariable interface
adapters for the following simple data types\\[2ex] 
\begin{tabular}{rrrrr}
float & int8 & int16 & int32\\
double & uint8 & uint16 & uint32\\
\end{tabular}\\[1ex]

It is assumed that float, double, 16 bit integer and 32 bit integer are
supported by most control systems.
In case the short integers are not supported, the next larger
integer with range limitation is used. If no range limitations are available
in the control system, the maximum allowed integer for the shorter type is set
and reported back to the control system. 64 bit integers are currently not supported 
by the process variable adapter. They might be added later.
Control system specific data types are not supported.

\subsection{Strings}

In addition to the simple data types, strings are supported as process variables.
Depending on the control system implementation the string might have limited
length. If an input string is too long it will be truncated. The string has a
“truncated” flag which can be queried. It intentionally does not throw an
exception in order to avoid servers from crashing which run fine on other
control systems. In most cases there will be a truncated status message, which
can be tolerated. There might be, however, malfunctions if the complete string
is parsed automatically. In this case the parsing code will generate the error
rather than the server terminating with an exception. 

\subsection{Arrays}

ChimeraTK supports arrays of all supported simple data types.

\section{Requirements for the ProcessVariable Adapter}\label{section_process_variable_adapter}

The ProcessVariable class holds an instance of the control system variable of
the corresponding type. The control system usually implements a thread and
locking scheme for the communication. This is the reason why ProcessVariables
cannot be considered real time capable. 

The business logic, however, will probably have a real time thread. For this
reason it cannot be avoided to do one copy: From the lock-free variable used
in the business logic into the control system variable. To avoid additional
copying the ProcessVariable adapter does not contain an instance of the basic
data type (or array), but accessors directly work with the control system
variable. 
If the business logic is performing calculations it is recommended that 
it keeps a local copy and only synchronises with the ProcessVariable once.
Usually all input variables are synchronised at the beginning of the 
algorithm, all output variables at the end.

The business logic has to create instances of process variables without
knowing which particular implementation is needed. For this a factory pattern
is used. Each variable is identified by a unique name, which is used to
request creation of the variable from the factory. The actual address, under
which the variable is seen from the control system, is control system specific
and handled in the concrete factory. It is determined from the unique name. 

The ProcessVariable adapter is not thread safe. If the underlying control
system variables are thread safe, then the adapter is also thread safe as it
does no further caching but directly accesses the underlying variable. But this
is not guaranteed by the interface and the user has to implement a thread safe
mechanism himself. ChimeraTK will provide thread support as a separate tool of
the Control System Tools package, probably implementing an actor pattern.

\subsection{The ProcessVariable adapter interface features}
\begin{itemize}
\item a get() method. It allows to register a callback function which is executed before
  the variable content is returned.
\item a set() method which triggers the ``on set'' callback functions
\item methods to set the ``on set'' and ``on get'' callback functions which are executed 
  if the corresponding accessors are called
\item set() and get() methods which does not trigger the callback functions. Like this
  it is possible the execute the callback functions only when the variable
  content is changed externally (from the control system), but update the
  variable from inside the business logic to inform the control system about
  the new value without triggering the callbacks. 
  \fixme{Sebastian proposed that the onGet callback should have the current value as parameter, 
    but I forgot what the use case was. Do we need it? MK}
\item an assignment operator (for convenience, like set function without
  callback)
\item a const implicit conversion operator. This operator does not trigger the ``on get'' callback
  function.
\item \fixme{Do we want range limits and engineering units?}
\end{itemize}
\fixme{It seems like get function and implicit conversion operator have to make a copy
  because it a priory cannot be guaranteed that the control system type allows
  access by reference to the underlying simple data type instance. Or should we implement 
  reference access where possible and hide the copy as implementation detail where required?}

\subsection{Additional feature of the array interfaces}
It \textcolor{red}{would be} desirable if the array had an interface compatible 
with a boost::array/std::array.
\begin{itemize}
\item {[ ]} operator to access individual elements
\item iterators
\item size()
\item front()
\item back()
\item empty()
\item swap() \fixme{Do we need it? Might not be possible to implement this efficiently. What to do if sizes are different? What is happening to the iterators (std::vector::swap also swaps the iterators, array does not)?}
\item fill()
\item a set() and setWithoutCallback() method for std::vectors
\item a get() and getWithoutCallback() method for std::vectors
\end{itemize}
The std::vector is the container to be used in the business logic. It allows dynamic allocation at
start-up (size might not be known at compile time) and opposes no overhead compared to
a raw buffer if the size is not changed.

However, with control systems that do not have a global lock the random access operator{[ ]} does not
work well:
\begin{itemize}
\item It would have to lock/unlock the variable on each access, which has bad performance
\item It is not guaranteed that the array is not changed or read while the business logic has only
      partly modified it.
\end{itemize}
Hence the random access operator is NOT part of the adapter interface.

Iterators have a similar issue. They would have to hold the lock which can make then inefficient.
In addition, the abstraction/virtualisation of the iterator makes them inefficient, because each acces
requies a virtual function call. In a first implementation, iterators will not be present.\\
\fixme{One could acquire the lock on construction of the iterator, and release it in the destructor.
Accessing and incrementing would still be efficient.}\\
\fixme{One could have an \texttt{iterate} function, similar to \texttt{std::for\_each}, except that it 
only has the function to be executed as argument. Can one write reasonable business logic with it?
Could this be template code so we get away without virtual iterators?
Such a function would also solve the problem when to trigger the ``onChange'' callback. With iterators 
it would have to be on releasing the lock in the destructor of the last iterator instance.}

In the current implementation the get() and set() functions, which do just copy, are the only reasonable
implementations. Exactly what we wanted to avoid.

\fixme{What about units for the x-axis (interpretation as binned data)?}


\subsection{Implementation details}

\begin{itemize}
\item Process variables are not copy-constructible as they have to be
  registered with a unique name. They have a private copy constructor. 
\item The constructor is also private. Construction is done with a
  factory. The construction call takes a programme-wide unique name to
  identify the variable. With this the factory will be able to determine the
  required information to create the control system variable from a config
  file or data base.\\
N.B. Making the constructor private disables the possibility that the user
writes control-system dependent code. 
\item All arrays are implemented as fixed size during run time. The size is 
      determined at start time and the memory is allocated. It cannot be changed.
      In case not all values of the array have been written, a value \texttt{nValidElements}
      can be set. \fixme{Is this just a convenience number without further functionality in
      the interface? Should it be recommended that only the shortened array is 
      transferred by the control system, if possible?
      What about the iterator range? Would it also be shorter?}
\item Callback functions can be unregistered.
\item In case signed variables have to be implemented as unsigned in the
  control system, or vice versa, a reinterpret-cast is done in order not to
  lose information. In this case the negative range/ upper half of the dynamic
  range has to be reinterpreted by the user. An 'out of range' flag is set. 
\item If only single precision floating point numbers are provided by the
  control system, the double is converted loosing precision. A 'limited
  precision' flag is set (statically in the concrete implementation.)
\item \fixme{Does it make sense to implement the callback functions asynchronously? One can use 
boost::async and boost::future for this. DOOCS already implements a variable update as a separate thread.
Is this implementation dependent?}
\end{itemize}

\subsection{Large Data Objects}

Some controls systems use synchronous operations like Remote Procedure Calls
for data transmission. This is not efficient for large data objects or objects
which have a very high update rate. In some cases (DOOCS for instance) the
control system has a special mechanism for these data, other control systems
like EPICS 3 usually require an 'out of band' implementation which adds a
proprietary protocol. Some control systems, however, use asynchronous data
protocols and can handle large data objects and high update rates efficiently
(EPICS 4, Tango, Karabo, Tine(?) ).

As the decision whether a special channel is needed, and for which which data sizes, depends on
the control system, this has to be handled in the control system specific
ProcessVariable factory. For the business logic and in the interface the
notion of a special channel for high data rates does not exist.

\subsection{Time Stamps}
There should be the possibility to have a time stamp with each process variable in order to
synchronise them later. The time stamp should consist of
\begin{itemize}
  \item uint32\_t UNIX time (number of seconds since 1970-01-01 00:00:00 UTC)
  \item uint32\_t Number of nanoseconds
  \item 2 $\times$ uint32\_t Two indices for storing unique IDs, like run and event number, macro pulse number etc.
\end{itemize}
\fixme{DOOCS variables don't have this. We would lose functionality.}

\fixme{I think this would be a huge overhead to always have this with every variable. Especially
  simple data types would grow by at least a factor 3 (with the current time stamp proposal
 seconds + nanoseconds + two user indices it would be a factor 5).}

\fixme{Could we create a variable group which has one time stamp object? This would also
  allow to send these objects in one transfer, for instance to the DAQ, which will be much more
  efficient if the control system serialiser allows it.}

\fixme{DOOCS variables do not have a history by default. How to determine which variables 
  have a history? There are many applications which just want the current state of some
  hardware, without the need to track changes. No time stamp or history is needed here.}

\fixme{Find a better name for the user ``indices'' (currently index0 and index1. userInt?}

Proposal (MK): Have a TimeStamp decorator for the ProcessVariables and arrays. There is a flag or
separate function to get a variable with time stamp. When the variable has a time stamp,
the history is activated automatically (or do we need a separate flag? A config word with 
features to be activated by the control system?)

\subsection{The ProcessVariable Factory}

The factory allows the abstraction of the variable creation. Each user class
which is programmed as a control system independent server is handed a
reference or shared pointer of a concrete factory implementation. The factory
is not implemented as a singleton pattern to allow the use of multiple
factories simultaneously, for instance one for low bandwidth channels and one
for high bandwidth channels (DAQ). This would allow to use the same 'out of
band' factory for multiple control systems with slow, synchronous calls. The
factory which is handed to the business logic would be a composite of the low
and high bandwidth factory in this case. 
For asynchronous control systems with sufficient bandwidth in the protocol the
factory would just have one component. All this is transparent to the business
logic.

The decision which channels is treated out of band is done inside the concrete
factory. It might be that one control system has to tread a certain amount of
data out of band while another control system can do this with its internal
protocol, but needs out of band for a higher data rate. Or that system A can
handle high update rates well, but has to treat large data blocks differently,
while system B can cope with the large data, but not with the high update
rate.

The factory is automatically instantiating and registering all variables
requested by the business logic. The server/application specific code should
not have to deal with the individual variables. It should only instantiate the
correct factory and pass it to the instance of the business logic class.

\subsection{Control System Stub}
For the unit testing a factory for a control system stub is implemented.
Stub variables can be set and modified via the ProcessVariable adapter and from
the C++ application which hosts the stub, incl. the callback functions on
change, but no control system functionality like network communication or
history is available. This allows to implement and run the business logic
stand-alone, enables full testing and even the possibility to write a GUI.

Nice to have: The control system stub should also have Python bindings so PyQT
or simple control scripts can be used.

\subsubsection{Implementation Detail}
The stub should implement a global lock and check if it is held by the calling process.
Like this the business logic can test if it is handling the process variables correctly.
It it recommended that all business logic is tested against the control system stub.

\section{Requirements for the Control System Function Callbacks}
\label{section_function_callbacks}
There are tree ways how actions in the business logic are called by or
synchronised with the control system:
\begin{enumerate}
  \item Synchronous actions with each read and write operation of a 
    process variable. This is implemented by the ``on set'' and 
    ``on get'' callback functions of the ProcessVariable adapter
    (section~\ref{section_process_variable_adapter})
  \item Update functions which are triggered by the control system.
    This can either be a periodic function or a function triggered 
    by an outside event, like a trigger signal which is send by the
    control system.
  \item Synchronisation functions which are called within the user code.
    This might be needed if the use code is running its own thread which
    for instance might be triggered by the hardware and the business logic
    has to determine when the synchronisation is executed.
\end{enumerate}

In the first two cases the timing is fully controlled by the control system.
In the third option the timing execution of the synchronisation is triggered by
the business logic, but it can be blocked by the control system adapter if there is
ongoing communication. The
implementation of the control system adapter has to make sure that it is (thread)
safe to access the ProcessVariables within the registered functions. It has to lock 
a mutex which locks all variables which have been created by this instance of the factory.
(In DOOCS this is one ``Location''. All variables are locked together so the user code
can rely on a consistent state which does not change during execution).
If more than one ``Location'' \fixme{How do we name this DOOCS-independently?} is needed,
the business logic would just get several factories from the control system adapter.

\fixme{What is the interface to get a factory? My first idea was just getting a pointer, but it
seems multiple factories might be possible, and the name of the location has to be given to the
factory. A ``master factory'', which gives access to multiple ProcessVariableFactories?}

\fixme{For the process variable callbacks it would be desirable to execute them in a separate
  thread if they need a significant amount of time. But do we gain anything? We still have to
  hold the location mutex, so all other actions on the location are still blocked.
  Should we leave it to the user to implement a thread pool or actor pattern if this action can be
  done without holding the mutex? I think it depends on the user application.}

\subsection{Control System Triggered Functions}
The business logic can register two functions with the control system adapter:
\begin{itemize}
  \item \texttt{void periodicUpdateFunction()}
  \item \texttt{void triggeredUpdateFunction()}
\end{itemize}
The user must never call these functions. They are only to be called by
the control system, which locks and unlocks the corresponding mutex so 
the access to the process variables is safe.

\fixme{Should these function have a time stamp as parameter, so the business logic knows when it was
triggered and can add this to the data?}

\subsection{Functions Triggered by the Business Logic}
As described above, some control systems require to hold a global lock, hence the busniness logic
must not access any process variables without holding it. To assure this, the business logic must
request the execution of all functions which are accessing process variables. These functions are called
synchronisation functions because they will usually be executed at the beginning or the end of 
the business logic's main loop to synchronise with the outside world.

\begin{itemize}
  \item \texttt{void executeSyncronisationFunction(boost::function<void ()>)}
\end{itemize}
This does the locking and unlocking and assures that
it is safe to modify the process variables and that they are not being changed
from the outside.

The signature of the function is just \texttt{void(void)}. If any arguments are needed, this can be done
using boost::bind before giving the function to the control system adapter for execution. Usually it
should not be necessary to have return values. There typically will be multiple process variables which
are class members of the business logic class which has the synchronisation function as a method, so
they are availabe inside this function. If need be the process variables can be ``call by reference''
parameter.

\fixme{How does this look like if the control system class is a/ requires a state machine?
Is business logic which is not a state machine compatible? How do both machines talk to each other?
Do we need a state machine adapter? Should all business logic in ChimeraTK also be a state machine?}

\section{Further, General Requirements}
As we have to support Ubuntu 12.4, C++11 is not used because the C++11 support
is not mature enough in this distribution. Boost is used for the required
functionality like shared pointers etc.

\appendix
\section{DOOCS Variable Types}

Directly supported by the control system adapter in bold
\begin{multicols}{3}
\noindent D\_name \\
\textbf{D\_float}\\
D\_float\_ptr \\
\textbf{D\_floatarray}\\
\textbf{D\_double}\\
D\_double\_ptr \\
\textbf{D\_doublearray}\\
D\_gen\_sts \\
D\_sts \\
D\_status \\
D\_polynom \\
D\_hist \\
D\_histStat \\
D\_comment \\
\textbf{D\_string} (80 characters version)\\
D\_zmqstring \\
D\_ustr \\
D\_text \\
D\_xml \\
D\_alarm \\
D\_devinfo \\
D\_plotinfo \\
D\_plotinfo\_p \\
\textbf{D\_spectrum} \\
D\_ifff \\
D\_filter \\
D\_dig\_filter \\
D\_bit \\
D\_xyzs \\
D\_iiii \\
\textbf{D\_int}\\
D\_int\_ptr \\
D\_shortarray\\
D\_intarray\\
D\_longarray\\
D\_call \\
D\_alarm \\
D\_exec \\
D\_error \\
D\_xy \\
D\_floatmean/D\_doublemean \\
D\_floatrms/D\_doublerms \\
D\_floatsigma/D\_doublesigma \\
D\_floatema/D\_doubleema \\
\textbf{D\_byte\_array}\\
D\_mdfloat\_array \\
D\_image \\
D\_imagec
\end{multicols}

Not supported by DOOCS:
\begin{itemize}
\item int8, int16 $\rightarrow$ use int32
\item uint (all sizes) $\rightarrow$ use signed int with range check
\end{itemize}

Although D\_\textit{xxx}array seems like the obvious choice, it is almost never used in practice.
Usually a D\_spectrum is used because it can directly be plotted in a panel. ChimeraTK
will follow this approach and implement arrays as D\_spectrum in the DOOCS adapter.
\fixme{The D\_spectrum only exists for floats, which means it can only hold integers up to 24 bits. What to do with real 32 bit arrays?}

\section{EPICS Variable Types}
Directly supported by the control system adapter in bold
\begin{multicols}{3}
\noindent \textbf{DBF\_STRING}\\
\textbf{DBF\_INT}\\
\textbf{DBF\_SHORT}\\
\textbf{DBF\_FLOAT}\\
DBF\_ENUM\\
\textbf{DBF\_CHAR}\\
\textbf{DBF\_LONG}\\
\textbf{DBF\_DOUBLE}
\end{multicols}

Unsigned integer types are not fully supported, therefore the corresponding
signed type should be used and interpreted as unsigned. DBF\_STRING is
limited to 40 characters including the terminating null byte (so 39
characters effective size). Fixed-size arrays of all listed data-types
are also supported.

\section{FIXMEs}
Number of remaining FIXMEs: \arabic{nFixmes}

\end{document}
